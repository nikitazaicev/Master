\chapter{Introduction}

This chapter is an introduction that explains the goals and reasons for this work.

\subsection{Goal}

\gls{ml} and Neural Networks have shown to be extremely potent and versatile in solving vast variety of problems across different fields. One research field that has been popular and challenging in the last few years is \gls{co}. \gls{co} includes problems such as \gls{mis} and \gls{mwm}. Such problems can be viewed in the context of graphs. \gls{gnn} is a subclass of \gls{nn}s designed specifically for solving problems related to graphs, but there are some problems that have not yet been solved efficiently with \gls{gnn}s. 

The main goal for this project is to: "Find out whether a \gls{gnn} can outperform approximations algorithms such as greedy in solving \gls{mwm} problem. 

\subsection{Reason}

Previous section describes "what" is this work about. Now lets briefly discuss "why" do this in the first place. The idea is rather simple. There are 2 types of algorithms in general: exact and approximate. The exact algorithm for \gls{mwm} already exists, but the down side of such algorithms is their speed. The slower nature of exact algorithms is one of the reasons why approximation algorithms are used in some cases. The hope is that  \gls{gnn} can potantialy squeeze between the approximation and optimal algorithms in terms of both time and accuracy.

In short, the reason is: "There can be a machine learning based algorithm that gives better results than existing approximation algorithms"

\section{Project structure}

The rest of this document has the following structure:

\begin{enumerate}

\item Background:

This Chapter will briefly touch the history, applications and impact of machine learning and go through core concepts and ideas behind machine learning and neural networks. Relevant information about \gls{co} and relevant algorithms will also be included here.

\item Methodology and Data:

This Chapter will focus on how the research was done and also explain core concepts of \gls{nn}s and the general procedure of training a \gls{nn}. Then the specifics of this case will be discussed together with the main challenges. The progress made step by step and the reasoning behind changes and choices made along the way will be shown. The chapter will also analyze the data used for training the model and evaluating results along with justification for the chosen data.

\item Results:

This Chapter will focus on temporary results produced during the reseach and explain how these results affected the further decisions. Finally the chapter will be concluded by analyzing the final results and comparing them with the expectations.

\item Conclusion:

Here the conclusion for this project will be drawn regarding whether the results gave any meaningfull insight and what future work can be done for improvements.

\end{enumerate}

\subsection{Listings}
You can do listings, like in Listing~\ref{ListingReference}
\begin{lstlisting}[caption={[Short caption]Look at this cool listing. Find the rest in Appendix~\ref{Listing}},label=ListingReference]
$ java -jar myAwesomeCode.jar
\end{lstlisting}

You can also do language highlighting for instance with Golang:
And in line~\ref{LineThatDoesSomething} of Listing~\ref{ListingGolang} you can see that we can ref to lines in listings.

\begin{lstlisting}[caption={Hello world in Golang},label=ListingGolang,escapechar=|]
package main

import "fmt"

func main() {
    fmt.Println("hello world") |\label{LineThatDoesSomething}|
}

\end{lstlisting}

\subsection{\gls{git}}

The whole project can be seen here: \url{https://github.com/nikitazaicev/Master}